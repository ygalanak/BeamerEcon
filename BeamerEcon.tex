% Copyright 2021 by Yannis Galanakis <galanakis.gian@gmail.com>
% February 2021
% This template is based on the beamerfontNord by Junwei Wang
% !TeX program = xelatex

\documentclass[compress,aspectratio=169, table]{beamer}
% theme
\usetheme[style=light]{Nord}
\usepackage{YGbeamer}

% title slide
\title[ShortTitle]{Long Title}
\subtitle{Subtitle}
\author{Yannis Galanakis}
\institute{University of Kent}
\date{February 2021}

%-------------------------------------------------------------------------
\begin{document}

\begin{frame}[plain,noframenumbering]
  \maketitle
\end{frame}


\section{Appearance}

\begin{frame}[fragile]
  \frametitle{Usage}
  Simply include the following code in your preamble:

  \begin{lstlisting}[basicstyle = \ttfamily\small]
\usetheme{Nord}
  \end{lstlisting}

  \bigskip

  By default, the appearance is in dark theme, however you can actively choose a either a light or a
  dark theme.

  \begin{lstlisting}[basicstyle = \ttfamily\small]
\usetheme[style=light]{Nord}
\usetheme[style=dark]{Nord}
  \end{lstlisting}

\end{frame}

\subsection{Colors}

\begin{frame}{Defined Colors}{This is a subtitle}
  \begin{description}[Snow Storm]
  \item[Polar Night]
    \textcolor{NordDarkBlack}{NordDarkBlack} \quad \textcolor{NordBlack}{NordBlack}\\
    \textcolor{NordMediumBlack}{NordMediumBlack} \quad \textcolor{NordBrightBlack}{NordBrightBlack}
  \item[Snow Storm]
    \textcolor{NordWhite}{NordWhite} \quad \textcolor{NordBrighterWhite}{NordBrightestWhite}\\
    \textcolor{NordBrightestWhite}{NordBrightestWhite}
  \item[Forest]
    \textcolor{NordCyan}{NordCyan} \quad \textcolor{NordBrightCyan}{NordBrightCyan}\\
    \textcolor{NordBlue}{NordBlue} \quad \textcolor{NordBrightBlue}{NordBrightBlue}
  \item[Aurora]
    \textcolor{NordRed}{NordRed} \quad \textcolor{NordOrange}{NordOrange} \\
    \textcolor{NordYellow}{NordYellow} \quad \textcolor{NordGreen}{NordGreen} \\
    \textcolor{NordMagenta}{NordMagenta}
  \end{description}
\end{frame}

\subsection{Fonts}

\begin{frame}[fragile]{Recommended Free Fonts}
  \begin{description}[Selected Fonts]
  \item[Selected Fonts] recommended for this theme\\
    \begin{lstlisting}[basicstyle = \ttfamily\small]
\setmainfont{Yanone Kaffeesatz}
\setsansfont{Fira Sans}
\setmonofont{DejaVu Sans Mono}
    \end{lstlisting}
  \item[Download] {\small \url{https://www.fontsquirrel.com/}}
  \item[Install Fonts] {\small \url{https://www.google.com/get/noto/help/install/}}
  \item[Compilation] compile with \XeLaTeX~to use system-wide fonts
  \end{description}

\end{frame}


\subsection{Blocks}

\begin{frame}
  \frametitle{Blocks}
  \begin{block}{This is a Block}
    \[
      a^2 + b^2 = c^2
    \]
  \end{block}
  \begin{exampleblock}{This is an Example Block}
    \[
      E = m \cdot c^{2}
    \]
  \end{exampleblock}
  \begin{alertblock}{This is an Alert Block}
    \[
      e^{i\pi} + 1 = 0
    \]
  \end{alertblock}

  \centering
  \begin{minipage}{1.0\linewidth}
    \begin{block}{Horizontally-Aligned Block}
      \[
        \log xy = \log x + \log y
      \]
    \end{block}
  \end{minipage}
\end{frame}

\subsection{Items}

\begin{frame}{Items}
  Itemize
  \begin{itemize}
    \item item 1
    \item item 2
  \end{itemize}

  \bigskip

  Enumerate
  \begin{enumerate}
    \item item 1
    \item item 2
  \end{enumerate}
\end{frame}

\subsection{Figures}

\begin{frame}{UK Economy}
\vspace*{-3em}
  \begin{figure}
    \centering
    % UK Economy graph
	\begin{tikzpicture}
		\begin{axis}[
			%width = \textwidth,
			axis lines* = left,
			ybar,
			bar shift = 0pt,
			bar width = 2pt,
			ylabel={GDP annual growth rate},
			xlabel = Year,
			xticklabel style={rotate=90,anchor=east, font=\tiny},
			xtick=data,
			xlabel style={yshift=-.01cm},
			extra x ticks={1991, 2008, 2009},
	%		nodes near coords,
			grid=none,
			tick label style={/pgf/number format/assume math mode},
			enlarge x limits,
			xticklabel style={/pgf/number format/set thousands separator={}}, 
			]
			
	\addplot+[NordBlue, fill=NordBlue]  table [x=Year,y=GDP_gr, col sep=comma] {figures/UK_economy.csv};
	\addplot+[NordMagenta, fill=NordMagenta]  table [x=Year,y=GDP_d, col sep=comma] {figures/UK_economy.csv};
	\draw[dashed] (axis cs: 2009,-4.5) -- (axis cs:2009,4.5);	
		\end{axis}
	
		\begin{axis}[
	%	width = \textwidth, 
		hide x axis,
		axis y line* = right,
		ylabel near ticks, yticklabel pos=right,
		grid = none,
		xlabel style={yshift=-.01cm},
		tick label style={/pgf/number format/assume math mode},
		enlarge x limits,
		ylabel= Unemployment Rate
		]
	\addplot[very thick, NordCyan]  table [x=Year,y=Unempl, col sep=comma] {figures/UK_economy.csv};	
	\node[color=NordCyan] at (axis cs: 2002, 6) {\scriptsize UR, total};
		\end{axis}
	\end{tikzpicture}
  \end{figure}
\end{frame}

%thank you slide
\begin{frame}[plain,noframenumbering]
	\centering
	{\Huge Thank you!}
	
	\begin{minipage}{0.4\textwidth}\hspace*{2em}
		\qrcode[height=1in, level=M]{https://ygalanak.github.io/}%
	\end{minipage}%
	\hfill
	\begin{minipage}{0.4\textwidth}\hspace*{1em}
		{ \large
			\textcolor{NordBlue}{\faTwitter}  @YannisGalanakis \\
			
			\hspace*{1em} \textcolor{NordBlue}{\faEnvelope}  I.Galanakis@kent.ac.uk 
		}
	\end{minipage}%
\end{frame}

%--------------------------------------------------
\appendix
\backupbegin
\begin{frame}{Appendix I}
A backup slide

\end{frame}
\backupend

\end{document}

%%% Local Variables:
%%% mode: latex
%%% TeX-master: t
%%% TeX-engine: xetex
%%% End:
